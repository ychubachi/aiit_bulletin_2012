\documentclass[a4j,9pt,twoside,twocolumn]{jsarticle}

%%%%%%%%%%%%%%%%%%%%%%%%%%%%%%%%%%%%%%%%
%% オプション
%%%%%%%%%%%%%%%%%%%%%%%%%%%%%%%%%%%%%%%%
%%
%% ■ 図表のキャプションについて
%% 英語でキャプションを記載する場合は下記2行のコメントを外してください
% \renewcommand{\figurename}{Fig.}
% \renewcommand{\tablename}{Table.}

%%%%%%%%%%%%%%%%%%%%%%%%%%%%%%%%%%%%%%%%
%% パッケージ
%%%%%%%%%%%%%%%%%%%%%%%%%%%%%%%%%%%%%%%%
\usepackage{aiitbulletin} % AIIT紀要スタイルの読み込み
\usepackage{graphicx} % 図の取り込み
%% ここに必要なパッケージを追加してください
% \usepackage{amsmath}  % 数式用
% \usepackage{amsthm} % 数式用

%%%%%%%%%%%%%%%%%%%%%%%%%%%%%%%%%%%%%%%%
% 標題・著者名・概要等
%%%%%%%%%%%%%%%%%%%%%%%%%%%%%%%%%%%%%%%%
\title{産技大紀要フォーマットについて}
\author{
産 技 太 郎
\thanks{産業技術大学院大学 産業技術研究科, School of Industrial Technology, Advanced Institute of Industrial Technology}
・
産 技 花 子
\thanks{首都大学東京,システムデザイン研究科, Graduate School of System Design, Tokyo
 Metropolitan University}
}
\etitle{Style and Layout of an AIIT Bulletin}
\eauthor{Taro Sangi\thanksmark{2} and Hanako Sangi\thanksmark{3}}
\eabstract{
Lorem ipsum dolor sit amet, consectetur adipiscing elit. Aenean sed
nulla nisi. Pellentesque habitant morbi tristique senectus et netus et malesuada
fames ac turpis egestas. Aliquam erat volutpat. Sed tristique ipsum eu nisl
viverra a accumsan elit ultricies. Fusce porta eros a lectus scelerisque vitae
eleifend odio interdum. Pellentesque habitant morbi tristique senectus et netus
et malesuada fames ac turpis egestas. Aliquam erat volutpat.
%
Ut pulvinar, massa vitae gravida pharetra, libero tellus tristique lectus, in
congue quam lectus in justo. Nulla pellentesque commodo lorem quis dictum. Nulla
lacus nulla, dapibus ac placerat sed, aliquam sit amet eros. Morbi eu
elit.(approx. 100 word)}
\keywords{Keywords in English. (approx. 5 Keywords)}
\receivedon{September 25, 2012} % 送付日

%%%%%%%%%%%%%%%%%%%%%%%%%%%%%%%%%%%%%%%%%%%%%%%%%%%%%%%%
%% 文書
%%%%%%%%%%%%%%%%%%%%%%%%%%%%%%%%%%%%%%%%%%%%%%%%%%%%%%%%
\begin{document}
% 以下の3行は標題・送付日等を印字するためのものです(編集しないでください)
\pagestyle{empty}
\maketitle\thispagestyle{empty} % 改行を入れないでください
\makereceivedon % 送付日を印字します

% %%%%%%%%%%%%%%%%%%%%%%%%%%%%%%%%%%%%%%%
% 本文
% %%%%%%%%%%%%%%%%%%%%%%%%%%%%%%%%%%%%%%%
\section{はじめに}
本稿では産業技術大学院大学紀要のフォーマットについて記す.
\LaTeX のフォーマットを使う執筆者はこのファイルの設定を変えずに,
このファイルの中身を書き換えて使うこと.

なお文中の「Lorem ipsum\ldots」から始まる文はダミーのテキストであるので読み飛ばすこと.

\section{標題等について}
\subsection{原稿}
原稿は,日本語もしくは英語による完全版下(camera ready)原稿とする.製版後の校正は原則として不可能であるため,誤字や脱字がないよう,特に念を入れて仕上げる.刷り上がりは,6頁以上が望ましい.

\subsection{標題}
標題は和文ならびに英文とする.英文原稿の場合は,和文表題を記述するところに英文標題を記述し,通常の英文標題のところは削除すること.

\subsection{著者名・所属}
著者名も英文による原稿の場合は,通常の和文著者名のところに英文で記述し,英文著者名のところは削除すること.所属も英文で記述すること.

\subsection{アブストラクト・キーワード}
和文ではなく英文で記述すること.アブストラクトは100語程度とし,キーワードは5つ程度とする.

\subsection{標題等の割付}
別紙産技大紀要カメラレディ用見本に従って,[和文標題,和文著者名,英文標
題,英文著者名,英文アブストラクト,英文キーワード,所属]の割付を行う.

\section{本文について}

\subsection{余白}
天地左右余白(マージン)・段間余白(コラムスペース)も産技大紀要カメラレディ用見本に従う.

\subsection{見出し}
原稿には,大見出し,中見出しなどを設け,それらを明瞭に区分する.さらに細分を要するときは,著者の分類に委ねる.

\subsection{句読点}
句読点には,全角ピリオド(.),全角コンマ(,)を用いること.

Lorem ipsum dolor sit amet, consectetur adipiscing elit. Aenean sed
nulla nisi. Pellentesque habitant morbi tristique senectus et netus et malesuada
fames ac turpis egestas. Aliquam erat volutpat. Sed tristique ipsum eu nisl
viverra a accumsan elit ultricies. Fusce porta eros a lectus scelerisque vitae
eleifend odio interdum. Pellentesque habitant morbi tristique senectus et netus
et malesuada fames ac turpis egestas. Aliquam erat volutpat.

Lorem ipsum dolor sit amet, consectetur adipiscing elit. Aenean sed
nulla nisi. Pellentesque habitant morbi tristique senectus et netus et malesuada
fames ac turpis egestas. Aliquam erat volutpat. Sed tristique ipsum eu nisl
viverra a accumsan elit ultricies. Fusce porta eros a lectus scelerisque vitae
eleifend odio interdum. Pellentesque habitant morbi tristique senectus et netus
et malesuada fames ac turpis egestas. Aliquam erat volutpat.

\section{図・表について}
\subsection{キャプション}
図・表には,図\ref{fig:VFS},図2,
表\ref{tab:cost2},表2 のように論文全体で通し番号をつけること.
表のキャプションは表の上に,図のキャプションは図の下につけること.図・表
ともに配置は中央揃えにすること.

英文の場合には,Fig. 1,Fig. 2,Table 1,
Table 2のように,番号をつけること.通し番号,標題は本文と同じ書体を使用
すること.

%%%%%%%%%%%%%%%%%%%%%%%%%%%%%%%%%%%%%%%%%%%%%%%%%%%%%
%%%                                               %%%
%%%                     Fig                       %%%
%%%                                               %%%
%%%%%%%%%%%%%%%%%%%%%%%%%%%%%%%%%%%%%%%%%%%%%%%%%%%%%
\begin{figure}[h]
\centering
\includegraphics[width=20zw]{aiit_logo.eps}
\caption{図のキャプション}
\label{fig:VFS}
\end{figure}
%%%%%%%%%%%%%%%%%%%%%%%%%%%%%%%%%%%%%%%%%%%%%%%%%%%%%
%%%                                               %%%
%%%                    Table                      %%%
%%%                                               %%%
%%%%%%%%%%%%%%%%%%%%%%%%%%%%%%%%%%%%%%%%%%%%%%%%%%%%%
\begin{table}[h]
\caption{表のキャプション}
\label{tab:cost2}
\centering
\begin{tabular}{|c|c|c|} \hline
A & B & C \\ \hline
$A_{1}$ & $B_{1}$ & $C_{1}$ \\ \hline
$A_{2}$ & $B_{2}$ & $C_{2}$ \\ \hline
$A_{3}$ & $B_{3}$ & $C_{3}$ \\ \hline
$A_{4}$ & $B_{4}$ & $C_{4}$ \\ \hline
\end{tabular}
\end{table}

\subsection{図の中の文字等}
 図・表は,印刷に十分耐えうるものでなければならない.刷り上がり時の文字が小さすぎないよう十二分に配慮し,線の太さにも注意する.

\subsection{色刷り}
図・表に色刷りを必要とする場合は,別途連絡すること.ただし,製本上の都合で色刷り頁を設けることができない場合もありうる.

\section{参考文献について}
%###
参考文献は,通し番号とし,本文中では,当該事項または人名などの参考とする
後に,\cite{TRA96HuHaCo},\cite{SICE02Yo}--\cite{Asa02Ar}
のように記す.文章の末尾に記す必要がある場合には,句読点の前に記す.参考文献は,原則として,雑誌の場合は,著者,標題,雑誌名,巻,号,頁,年の順に記す.また,著書の場合は,著者,書名,発行所,発行年の順に記す.参考文献例を本文の最後に挙げるので参考されたい.
%###

\section{おわりに}
本稿では産業技術大学院大学紀要のフォーマットについて記した.

Lorem ipsum dolor sit amet, consectetur adipiscing elit. Aenean sed
nulla nisi. Pellentesque habitant morbi tristique senectus et netus et malesuada
fames ac turpis egestas. Aliquam erat volutpat. Sed tristique ipsum eu nisl
viverra a accumsan elit ultricies. Fusce porta eros a lectus scelerisque vitae
eleifend odio interdum. Pellentesque habitant morbi tristique senectus et netus
et malesuada fames ac turpis egestas. Aliquam erat volutpat.

Lorem ipsum dolor sit amet, consectetur adipiscing elit. Aenean sed
nulla nisi. Pellentesque habitant morbi tristique senectus et netus et malesuada
fames ac turpis egestas. Aliquam erat volutpat. Sed tristique ipsum eu nisl
viverra a accumsan elit ultricies. Fusce porta eros a lectus scelerisque vitae
eleifend odio interdum. Pellentesque habitant morbi tristique senectus et netus
et malesuada fames ac turpis egestas. Aliquam erat volutpat.

Lorem ipsum dolor sit amet, consectetur adipiscing elit. Aenean sed
nulla nisi. Pellentesque habitant morbi tristique senectus et netus et malesuada
fames ac turpis egestas. Aliquam erat volutpat. Sed tristique ipsum eu nisl
viverra a accumsan elit ultricies. Fusce porta eros a lectus scelerisque vitae
eleifend odio interdum. Pellentesque habitant morbi tristique senectus et netus
et malesuada fames ac turpis egestas. Aliquam erat volutpat.

Lorem ipsum dolor sit amet, consectetur adipiscing elit. Aenean sed
nulla nisi. Pellentesque habitant morbi tristique senectus et netus et malesuada

\begin{thebibliography}{99}
\bibitem{TRA96HuHaCo} 
S. Hutchinson, G. D. Hager and P. I. Corke,
``A Tutorial on Visual Servo Control,''
{\it IEEE Trans. Robotics and Automation},
Vol.~12, No.~5, pp.~651--670, 1996.
%
\bibitem{SICE02Yo}
吉川恒夫,
``ロボット技術,''
計測と制御, Vol.~41, No.~1, pp.~17--21, 2002.
%
\bibitem{Joh06SpHuVi} 
M. W. Spong, S. Hutchinson and M. Vidyasagar, 
{\it Robot Modeling and Control}, 
John Wiley \& Sons, 2006. 
%
\bibitem{Asa02Ar}
有本 卓, 新版 ロボットの力学と制御, 朝倉書店, 2002. 
\end{thebibliography}

\end{document}


