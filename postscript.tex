\documentclass[a4j,14Q, twoside]{jsarticle}

\title{編 集 後 記}
\author{紀要編集委員長\\中鉢 欣秀}
\date{}

\begin{document}
\pagestyle{empty}
\maketitle

この度,産業技術大学院大学紀要第6号を発刊する運びとなった.本学の教員を
中心とする著者による28件の論文を収録し,多様な研究成果がこの中には含ま
れている.情報科学や機械工学の基礎的研究もあれば情報システムの応用,な
いしは,デザイン・感性領域に関するものもある.分野融合型の新しい研究を
試みる本学の特性が,この紀要という形で改めて顕示されたと言えよう.

著者から投稿された論文は編集委員による査読を経て掲載しており,各論文の
種別は,著者から送付された投稿票に記載されたものをそのまま掲載している.
ただし,本学における研究の多様性から,査読にあたっては一般的な学術論文
誌のような内容面の評価は困難であり,編集委員会では明らかな記述の誤り・
誤字脱字などの有無を確認することに留めるものとした.これは,これらの多
様な研究成果を一律な評価軸で判断するのではなく,むしろその多様性を保っ
たまま一つの論文誌という形に纏めるということこそ本学紀要の意義なのでは
ないかとの考えに基づく.

紀要の編集方針は,歴代の編集担当者の見解を踏まえ様々な変遷を辿ってきた
と聞く.紀要の編集のあり方については今後も継続的に議論をする必要があろ
う.例えば,他分野の読者が研究の内容を評価できるような評価軸を提示する
ことを予め著者に求めたり,研究の過程における投稿論文の位置付けについて
より明確に記載してもらったり,といった工夫も必要なのかもしれない.

今回,編集委員長としてすべての論文に目を通す機会が与えられた.私自身は
情報システムが専門であるが,創造系の教員の論文も大変興味深く拝読させて
いただいた.逆に,情報系でも相当に高度なご研究もあって,なかなか読み応
えがあった.

本誌を手に取られた方には,他の分野の研究者の手による論文もぜひご一読し
ていただきたいと切に願う.他分野との融合から新しい果実を得るためには,
自分と異なる立場に対する理解が不可欠であり,そのために必要となる努力を
し続けることこそが,次世代への飛躍につながるのだと思う.

末筆ながら,論文を執筆された著者の皆様,編集委員各位,及び,本学事務局
に感謝を申し上げ,本稿の締めとしたい.


\end{document}
