\documentclass[a4j,twoside]{jarticle}
\usepackage{kiyou}
\usepackage[dvipdfmx]{graphicx}

\title{編 集 後 記}
\author{紀要編集委員長 中鉢欣秀}
\date{}

\begin{document}
\pagestyle{empty}
\maketitle
%\begin{center}
%\fontsize{17pt}{17pt}\selectfont
%%% 論文標題
%  編 集 後 記
%\vskip\baselineskip
%\fontsize{12pt}{12pt}\selectfont
%%% 和文著者名
%  紀要編集委員長 中鉢欣秀
%\end{center}

この度,産業技術大学紀要第6号を発刊する運びとなった.本学の教員を中心と
する著者による28件の論文を収録し,多様な研究成果(情報科学や機械工学の
基礎的研究もあればデザイン・感性領域に関するものもある)がこの中には含
まれている.分野融合型の新しい研究を試みる本学の特性が,この紀要という
形で改めて顕示されたと言えよう.

著者から投稿された論文は編集委員による査読を経て掲載しており,各論文の
種別は,著者から送付された投稿票に記載されたものをそのまま掲載している.
ただし,本学の研究者の多様性から,査読にあたっては一般的な学術論文誌の
ような内容面の評価は困難であったため,編集委員会では明らかな記述の誤り・
誤字脱字などの有無を確認することに留めた.これは,これらの多様な研究成
果を一律な評価軸で判断するのではなく,むしろその多様性を保ったまま一つ
の論文集という形に纏めるということこそ本学紀要の意義なのではないかと考
たからである.

本学の紀要の編集にあたっては歴代の編集担当者の見解を踏まえ,様々な変遷
をたどっていると聞く.紀要の編集のあり方については今後も議論をする必要
があろう.例えば,他分野の読者が研究の内容を評価できるような評価軸を提
示することを著者に予め依頼したり,研究論文の位置付けについてより明確に
記載してもらったり,といった対応も必要なのかもしれない.

今回,編集委員長としてすべての論文に目を通す機会を与えていただいた.私
自身は情報システムが専門であるが,創造系の教員の論文も大変興味深く拝読
させていただいた.逆に,情報系でも相当に高度なご研究もあり,なかなか読
み応えがあった.

本誌を手に取られた方には,他の分野の研究者の手による論文もぜひご一読し
て頂きたいと切に願う.他分野との融合から新しい果実を生み出すためには,
自分と異なる立場に対する理解が必要であり,そのために必要となる努力は惜
しまないことこそが次世代への飛躍につながるものと個人的には思っている.

末筆ながら,本紀要を取りまとめるにあたり論文を執筆された著者の皆様,編
集委員各位,及び,産業技術大学院大学事務局に感謝を申し上げ,本稿を締め
くくることにする.


\end{document}
