%% ※このファイルは修正しないでください※

%% TODO: 和文の標題と著者名を表示しない方法

\documentclass[a4j,9pt,twoside]{jsarticle}
%%%%%%%%%%%%%%%%%%%%%%%%%%%%%%%%%%%%%%%%%%%%%%%%%%%%%%%%
%% パッケージ
%%%%%%%%%%%%%%%%%%%%%%%%%%%%%%%%%%%%%%%%%%%%%%%%%%%%%%%%
\usepackage{multicol} % 2段組み用(必須)

%%%%%%%%%%%%%%%%%%%%%%%%%%%%%%%%%%%%%%%%%%%%%%%%%%%%%%%%
%% ページのサイズ
%%%%%%%%%%%%%%%%%%%%%%%%%%%%%%%%%%%%%%%%%%%%%%%%%%%%%%%%
% 左右マージン
\setlength{\oddsidemargin}{-1.4mm}    % 標準(1in=25.4mm) -1.4mm = 24.0mm
\setlength{\evensidemargin}{-5.4mm}   % 標準(1in=25.4mm) -5.4mm = 20.0mm
% 上下マージン
\setlength{\topmargin}{4.6mm}         % 標準(1in=25.4mm) +4.6mm = 30.0mm
\setlength{\headheight}{0.0mm}
\setlength{\headsep}{0.0mm}
% テキストの大きさ
% (jsarticleで9pt指定のため長さの単位にtruemmを使っています)
\setlength{\textwidth}{166.0truemm}   % A4横210.0mm -(24.0+20.0)mm = 166.0mm
\setlength{\textheight}{247.0truemm}  % A4縦297mm - (20.0 + 30.0)mm =247.0mm

  
%%%%%%%%%%%%%%%%%%%%%%%%%%%%%%%%%%%%%%%%%%%%%%%%%%%%%%%%
%% 明朝体イタリックフォントに関する警告の抑制
%%%%%%%%%%%%%%%%%%%%%%%%%%%%%%%%%%%%%%%%%%%%%%%%%%%%%%%%
\DeclareFontShape{JY1}{mc}{m}{it}{<5> <6> <7> <8> <9> <10> sgen*min
    <10.95><12><14.4><17.28><20.74><24.88> min10 <-> min10}{}
\DeclareFontShape{JT1}{mc}{m}{it}{<5> <6> <7> <8> <9> <10> sgen*tmin
    <10.95><12><14.4><17.28><20.74><24.88> tmin10 <-> tmin10}{}

%%%%%%%%%%%%%%%%%%%%%%%%%%%%%%%%%%%%%%%%%%%%%%%%%%%%%%%%
%% タイトルの定義
%%%%%%%%%%%%%%%%%%%%%%%%%%%%%%%%%%%%%%%%%%%%%%%%%%%%%%%%
\makeatletter% @記号を通常文字として扱う
\newcommand*{\etitle}[1]{\gdef\@etitle{#1}}
\newcommand*{\eauthor}[1]{\gdef\@eauthor{#1}}
\newcommand*{\eabstract}[1]{\long\def\@eabstract{#1}}
\newcommand*{\keywords}[1]{\gdef\@keywords{#1}}
% http://ea3pch.yz.yamagata-u.ac.jp/member/sumio/tex/styleuse.pdf
\renewcommand{\maketitle}{%
  %\newpage\null
  %\vskip 2em
  \begin{center}
    \let\footnote\thanks
    % 和文標題
    {\LARGE \@title \par}%
    \vskip 1.5em
    {\large \@author \par}
    \vskip 2em
    % 英文標題
    {\Large\bf \@etitle \par}%
    \vskip 1em
    {\large\bf \@eauthor \par}%
    \vskip 2em
    % 英文概要
    {\large\bf Abstract}
  \end{center}%
  {\normalsize \@eabstract \par}
  \vskip 1em
  {\normalsize {\bf Keywords:} \@keywords \par} 
  \vskip 2em
  \setcounter{footnote}{0}%
  \global\let\thanks\relax
  \global\let\maketitle\relax
  \global\let\@thanks\@empty
  \global\let\@title\@empty
  \global\let\@author\@empty
  \global\let\@etitle\@empty
  \global\let\@date\@empty
  \global\let\title\relax
  \global\let\author\relax
  \global\let\date\relax
  \global\let\and\relax
}%
\makeatother%  @記号を特殊文字に戻す

%%%%%%%%%%%%%%%%%%%%%%%%%%%%%%%%%%%%%%%%
%% 図表番号の定義
%%%%%%%%%%%%%%%%%%%%%%%%%%%%%%%%%%%%%%%%

\makeatletter% @記号を通常文字として扱う
\newcommand{\figcaption}[1]{\def\@captype{figure}\caption{#1}}
\newcommand{\tblcaption}[1]{\def\@captype{table}\caption{#1}}
\long\def\@makecaption#1#2{% キャプションの書式を定義します
  \vskip\abovecaptionskip
  \sbox\@tempboxa{#1: #2}% 例)「図1: キャプション」(コロンを補っています)
  \ifdim \wd\@tempboxa >\hsize
    #1: #2\par
  \else
    \global \@minipagefalse
    \hb@xt@\hsize{\hfil\box\@tempboxa\hfil}%
  \fi
  \vskip\belowcaptionskip}
\makeatother%  @記号を特殊文字に戻す

%%%%%%%%%%%%%%%%%%%%%%%%%%%%%%%%%%%%%%%%
%% セクションの定義
%%%%%%%%%%%%%%%%%%%%%%%%%%%%%%%%%%%%%%%%
\makeatletter% @記号を通常文字として扱う
% セクションの項目番号の書式を設定します
\renewcommand{\thesection}{\arabic{section}.}
\renewcommand{\thesubsection}{\arabic{section}.\arabic{subsection}.}
\renewcommand{\thesubsubsection}{\arabic{section}.\arabic{subsection}.\arabic{subsubsection}.}
% セクションの書式を設定します
\renewcommand{\section}{\@startsection{section}{1}{\z@}%
   {0\Cvs}%{0.5\Cvs \@plus1.0\Cvs \@minus0.5\Cvs}%
   {0\Cvs}%{0.5\Cvs \@plus1.0\Cvs \@minus.5\Cvs}%
   {\reset@font\large\bfseries}}
\renewcommand{\subsection}{\@startsection{subsection}{2}{\z@}%
   {0\Cvs}%{0\Cvs \@plus.0\Cvs \@minus.0\Cvs}%
   {0\Cvs}%{0\Cvs \@plus.0\Cvs \@minus.0\Cvs}%
   {\reset@font\normalsize\bfseries}}
\renewcommand{\subsubsection}{\@startsection{subsubsection}{3}{\z@}%
   {0\Cvs}%{0\Cvs \@plus.0\Cvs \@minus.0\Cvs}%
   {0\Cvs}%{0\Cvs \@plus.0\Cvs \@minus.0\Cvs}%
   {\reset@font\normalsize\bfseries}}
\makeatother% @記号を特殊文字に戻す

%%%%%%%%%%%%%%%%%%%%%%%%%%%%%%%%%%%%%%%%%%%%%%%%%%%%%%%%
%% 参考文献
%%%%%%%%%%%%%%%%%%%%%%%%%%%%%%%%%%%%%%%%%%%%%%%%%%%%%%%%
\renewcommand{\refname}{\centering \normalsize \bf 参考文献}

