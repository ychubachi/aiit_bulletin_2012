%% 産業技術大学院大学紀要
%% multicolを利用しているため\captionが使えません.
%% 代わりに,\figcaption,\tblcaption を用いて
%% 図と表の caption をつけて下さい.
\documentclass[a4j,9pt,twoside]{jarticle}

%%%%%%%%%%%%%%%%%%%%%%%%%%%%%%%%%%%%%%%%
%% パッケージ
%%%%%%%%%%%%%%%%%%%%%%%%%%%%%%%%%%%%%%%%
\usepackage{graphicx}
% \usepackage{latexsym}
\usepackage{amsmath}  % 数式用
\usepackage{amsthm}
\usepackage{multicol}
\usepackage{fancyhdr} % ヘッダー・フッター用
\usepackage{newcent}

%%%%%%%%%%%%%%%%%%%%%%%%%%%%%%%%%%%%%%%%
%% 図表番号の定義
%%%%%%%%%%%%%%%%%%%%%%%%%%%%%%%%%%%%%%%%

%% 英語でキャプションを記載する場合は下記2行のコメントを外してください
% \renewcommand{\figurename}{Fig.}
% \renewcommand{\tablename}{Table.}

\makeatletter% @記号を通常文字として扱う
\newcommand{\figcaption}[1]{\def\@captype{figure}\caption{#1}}
\newcommand{\tblcaption}[1]{\def\@captype{table}\caption{#1}}
\long\def\@makecaption#1#2{% キャプションの書式を定義します
  \small%% add this line
  \vskip\abovecaptionskip
  \sbox\@tempboxa{#1: #2}% 例)「図1: キャプション」(コロンを補っています)
  \ifdim \wd\@tempboxa >\hsize
    #1: #2\par
  \else
    \global \@minipagefalse
    \hb@xt@\hsize{\hfil\box\@tempboxa\hfil}%
  \fi
  \vskip\belowcaptionskip}
\makeatother%  @記号を特殊文字に戻す

%%%%%%%%%%%%%%%%%%%%%%%%%%%%%%%%%%%%%%%%
%% セクションの定義
%%%%%%%%%%%%%%%%%%%%%%%%%%%%%%%%%%%%%%%%
\makeatletter% @記号を通常文字として扱う
% セクションの項目番号の書式を設定します
\renewcommand{\thesection}{\arabic{section}.}
\renewcommand{\thesubsection}{\arabic{section}.\arabic{subsection}.}
\renewcommand{\thesubsubsection}{\arabic{section}.\arabic{subsection}.\arabic{subsubsection}.}
% セクションの書式を設定します
\renewcommand{\section}{\@startsection{section}{1}{\z@}%
   {1.5\Cvs \@plus.5\Cvs \@minus.2\Cvs}%
   {.3\Cvs \@plus.3\Cvs}%
   {\reset@font\normalsize\bfseries}}
\renewcommand{\subsection}{\@startsection{subsection}{2}{\z@}%
   {1\Cvs \@plus.5\Cvs \@minus.2\Cvs}%
   {.01\Cvs \@plus.01\Cvs}%
   {\reset@font\small\bfseries}}
\renewcommand{\subsubsection}{\@startsection{subsubsection}{3}{\z@}%
   {1\Cvs \@plus.5\Cvs \@minus.2\Cvs}%
   {.01\Cvs \@plus.01\Cvs}%
   {\reset@font\small\bfseries}}
\makeatother% @記号を特殊文字に戻す

%%%%%%%%%%%%%%%%%%%%%%%%%%%%%%%%%%%%%%%%%%%%%%%%%%%%%%%%
%% ページのサイズ
%%%%%%%%%%%%%%%%%%%%%%%%%%%%%%%%%%%%%%%%%%%%%%%%%%%%%%%%
\setlength{\oddsidemargin}{-1.4mm}
\setlength{\evensidemargin}{-5.4mm}
\setlength{\textwidth}{16.6cm}
\setlength{\topmargin}{-7.4mm}
\setlength{\headsep}{4.5mm}
\setlength{\textheight}{24.5cm}

%%%%%%%%%%%%%%%%%%%%%%%%%%%%%%%%%%%%%%%%%%%%%%%%%%%%%%%%
%% 明朝体イタリックフォントに関する警告の抑制
%%%%%%%%%%%%%%%%%%%%%%%%%%%%%%%%%%%%%%%%%%%%%%%%%%%%%%%%
\DeclareFontShape{JY1}{mc}{m}{it}{<5> <6> <7> <8> <9> <10> sgen*min
    <10.95><12><14.4><17.28><20.74><24.88> min10 <-> min10}{}
\DeclareFontShape{JT1}{mc}{m}{it}{<5> <6> <7> <8> <9> <10> sgen*tmin
    <10.95><12><14.4><17.28><20.74><24.88> tmin10 <-> tmin10}{}

%%%%%%%%%%%%%%%%%%%%%%%%%%%%%%%%%%%%%%%%%%%%%%%%%%%%%%%%
%% ヘッダー・フッター
%%%%%%%%%%%%%%%%%%%%%%%%%%%%%%%%%%%%%%%%%%%%%%%%%%%%%%%%
\pagestyle{fancy}
\lhead[{\small \thepage\hspace{2zw}
%ヘッダー・著者名 
中鉢欣秀
}]{}
\chead[]{} 
\rhead[]{{\small
%ヘッダー・論文タイトル
あああ産技大紀要フォーマットについて
\hspace{2zw}\thepage}}
\lfoot[]{} 
\cfoot[]{} 
\rfoot[]{} 
\renewcommand{\headrulewidth}{0.4pt} % ヘッダー罫線

%%%%%%%%%%%%%%%%%%%%%%%%%%%%%%%%%%%%%%%%%%%%%%%%%%%%%%%%
%% 参考文献
%%%%%%%%%%%%%%%%%%%%%%%%%%%%%%%%%%%%%%%%%%%%%%%%%%%%%%%%
\renewcommand{\refname}{{\normalsize \bf 参考文献}}


%%%%%%%%%%%%%%%%%%%%%%%%%%%%%%%%%%%%%%%%%%%%%%%%%%%%%%%%
%% 文書
%%%%%%%%%%%%%%%%%%%%%%%%%%%%%%%%%%%%%%%%%%%%%%%%%%%%%%%%
\begin{document}
\renewcommand{\thefootnote}{\fnsymbol{footnote}}{\fnsymbol{footnote}}

%%%%%%%%%%%%%%%%%%%%%%%%%%%%%%%%%%%%%%%%
%% 標題(1段組み)
%%%%%%%%%%%%%%%%%%%%%%%%%%%%%%%%%%%%%%%%
\vspace{-0.78cm}

\begin{center}
\vspace{0.5cm}

{\LARGE \bf 
% 論文タイトル
あああ産技大紀要フォーマットについて
}

\vspace{.6cm}

{\large 
% 論文著者名
中 鉢 欣 秀$^{*}$
}

\vspace{.6cm}

{\Large 
% 論文英語タイトル
Co-creative Software Development
}\\

\vspace{.4cm}

{\large 
% 論文英語著者名
Yoshihide Chubachi$^{*}$
}

\vspace{.4cm}
{\large Abstract}
\end{center}

\vspace{-2ex}
\noindent
{\normalsize
% アブストラクト
Abstract in English. (approx. 100 Words) 
\vspace{.3cm}

\noindent
Keywords: 
% キーワード
Keywords in English. (approx. 5 Keywords)
}

\vspace{.5cm}

%%%%%%%%%%%%%%%%%%%%%%%%%%%%%%%%%%%%%%%%
%% 本文(2段組み)
%%%%%%%%%%%%%%%%%%%%%%%%%%%%%%%%%%%%%%%%
\begin{multicols}{2}

%%%%%%%%%%%%%%%%%%%%%%%%%%%%%%%%%%%%%%%%
%% 送付日・所属
%%%%%%%%%%%%%%%%%%%%%%%%%%%%%%%%%%%%%%%%
\small
\setlength{\skip\footins}{1.7cm}
\setlength{\baselineskip}{14.75pt}
\footnotetext{\hspace{-4.5ex}
% 送付日
Received on September 25, 2010.
}
\footnotetext{\hspace{-4.5ex}
% 著者所属 1
$^{*}$ 産業技術研究科, School of Industrial Technology, AIIT
}
\footnotetext{\vspace{-8mm}} % 消すな

%\pagestyle{empty}

%%%%%%%%%%%%%%%%%%%%%%%%%%%%%%%%%%%%%%%%%%%%%%%%%%%%%%%%
%% ここから本文
%%%%%%%%%%%%%%%%%%%%%%%%%%%%%%%%%%%%%%%%%%%%%%%%%%%%%%%%

\section{はじめに}
本稿では産業技術大学院大学紀要のフォーマットについて記す.
Tex のフォーマットを使う執筆者はこのファイルの設定を変えずに,
このファイルの中身を書き換えて使うこと.

\section{標題等ついて}
\subsection{原稿}
原稿は,日本語もしくは英語による完全版下(camera ready)原稿とする.製版後の校正は原則として不可能であるため,誤字や脱字がないよう,特に念を入れて仕上げる.刷り上がりは,6頁以上が望ましい.

\subsection{標題}
標題は和文ならびに英文とする.英文原稿の場合は,和文表題を記述するところに英文標題を記述し,通常の英文標題のところは削除すること.

\subsection{著者名・所属}
著者名も英文による原稿の場合は,通常の和文著者名のところに英文で記述し,英文著者名のところは削除すること.所属も英文で記述すること.

\subsection{アブストラクト・キーワード}
和文ではなく英文で記述すること.アブストラクトは100語程度とし,キーワードは5つ程度とする.

\subsection{標題等の割付}
別紙産技大紀要カメラレディ用見本に従って,[和文標題,和文著者名,英文標
題,英文著者名,英文アブストラクト,英文キーワード,所属]の割付を行う.


\section{本文について}

\subsection{余白}
天地左右余白(マージン)・段間余白(コラムスペース)も産技大紀要カメラレディ用見本に従う.

\subsection{見出し}
原稿には,大見出し,中見出しなどを設け,それらを明瞭に区分する.さらに細分を要するときは,著者の分類に委ねる.

\subsection{句読点}
句読点には,全角ピリオド(.),全角コンマ(,)を用いること.


\section{図・表について}
\subsection{キャプション}
図・表には,図1,図2,表1,表2 のように論文全体で通し番号をつけること.
表のキャプションは表の上に,図のキャプションは図の下につけること.図・表
ともに配置は中央揃えにすること.英文の場合には,Fig. 1,Fig. 2,Table 1,
Table 2のように,番号をつけること.通し番号,標題は本文と同じ書体を使用
すること.multicols を用いたため figure 環境,table 環境は使えない.
通常の caption が使えないため.代わりに figcaption,tblcaption を用いて
図と表の caption をつけること.以下に例を示す.
%%%%%%%%%%%%%%%%%%%%%%%%%%%%%%%%%%%%%%%%%%%%%%%%%%%%%
%%%                                               %%%
%%%                     Fig                       %%%
%%%                                               %%%
%%%%%%%%%%%%%%%%%%%%%%%%%%%%%%%%%%%%%%%%%%%%%%%%%%%%%
%\begin{center}
%\scalebox{.2}{\includegraphics{AIIT_Symbol.eps}}
%\figcaption{図のキャプション}
%\label{fig:VFS}
%\end{center}
%%%%%%%%%%%%%%%%%%%%%%%%%%%%%%%%%%%%%%%%%%%%%%%%%%%%%
%%%                                               %%%
%%%                    Table                      %%%
%%%                                               %%%
%%%%%%%%%%%%%%%%%%%%%%%%%%%%%%%%%%%%%%%%%%%%%%%%%%%%%
\tblcaption{表のキャプション}
\label{tab:cost2}
\begin{center}
\begin{tabular}{|c|c|c|} \hline
A & B & C \\ \hline
$A_{1}$ & $B_{1}$ & $C_{1}$ \\ \hline
$A_{2}$ & $B_{2}$ & $C_{2}$ \\ \hline
$A_{3}$ & $B_{3}$ & $C_{3}$ \\ \hline
$A_{4}$ & $B_{4}$ & $C_{4}$ \\ \hline
\end{tabular}
\end{center}

\subsection{図の中の文字等}
 図・表は,印刷に十分耐えうるものでなければならない.刷り上がり時の文字が小さすぎないよう十二分に配慮し,線の太さにも注意する.

\subsection{色刷り}
図・表に色刷りを必要とする場合は,別途連絡すること.ただし,製本上の都合で色刷り頁を設けることができない場合もありうる.

\section{参考文献について}
参考文献は,通し番号とし,本文中では,当該事項または人名などの参考とする
後に,\cite{TRA96HuHaCo},\cite{SICE02Yo}--\cite{Asa02Ar}
のように記す.文章の末尾に記す必要がある場合には,句読点の前に記す.参考文献は,原則として,雑誌の場合は,著者,標題,雑誌名,巻,号,頁,年の順に記す.また,著書の場合は,著者,書名,発行所,発行年の順に記す.参考文献例を本文の最後に挙げるので参考されたい.

\section{おわりに}
本稿では産業技術大学院大学紀要のフォーマットについて記した.



\begin{thebibliography}{99}
\bibitem{TRA96HuHaCo} 
S. Hutchinson, G. D. Hager and P. I. Corke,
``A Tutorial on Visual Servo Control,''
{\it IEEE Trans. Robotics and Automation},
Vol.~12, No.~5, pp.~651--670, 1996.
%
\bibitem{SICE02Yo}
吉川恒夫,
``ロボット技術,''
計測と制御, Vol.~41, No.~1, pp.~17--21, 2002.
%
\bibitem{Joh06SpHuVi} 
M. W. Spong, S. Hutchinson and M. Vidyasagar, 
{\it Robot Modeling and Control}, 
John Wiley \& Sons, 2006. 
%
\bibitem{Asa02Ar}
有本 卓, 新版 ロボットの力学と制御, 朝倉書店, 2002. 
\end{thebibliography}








\newpage % 消さないで

\end{multicols}
\end{document}


